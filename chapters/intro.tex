% !TEX root = ../thesis.tex

\chapter{Introduction}%
\label{cha:introduction}

The overall aim of this thesis is to use transaction-level data from a
financial aggregator app, insights from behavioural science, and methods from
econometrics and machine learning to contribute to a better understanding of
the determinants of how individuals spend their money and save for the
short-term. This introductory chapter first discusses the context within which
this research takes place -- the importance of financial wellbeing, relevant
insights from behavioural science, and opportunities presented by the emergence
of FinTech apps -- and then provides an overview of the main chapters of the
thesis.

\section{Financial wellbeing}%

Financial wellbeing is a state in which a person can make ends meet in the
present, can feel comfortable about their financial future, can feel
comfortable about money, and has the financial freedom to make choices that
allow for the enjoyment of life. This includes having control over one's
day-to-day and month-to-month finances, not having to borrow to meet
ongoing obligations, being free of debt or being in control of it, having the
capacity to meet unexpected expenses, and -- for a working-age individual --
being on track to build enough savings for retirement.\footnote{The US Consumer
    Financial Protection Bureau defines financial wellbeing as ``a state of
    being wherein a person can fully meet current and ongoing financial
    obligations, can feel secure in their financial future, and is able to make
    choices that allow enjoyment of life'' \citep{cfpb2015financial}. The UK's
    Money and Pension service defines current financial wellbeing as ``being
    able to pay the bills and feel comfortable about money'' and longer-term
    financial security as ``having the savings to deal with the expected, the
    unexpected and the longer-term; or having some form of loss protection such
    as home contents or life insurance'' \citep{mps2018building}. Both
    organisations also differentiate some aspects of financial wellbeing
    depending on whether an individual is in their working or retirement age.
Throughout, this discussion, as well as in the analysis in the following
chapters, I focus on working age individuals.}

Financial wellbeing is important because a lack of it can have severe
consequences both in the short-term and in the long-term. In the short-term,
low financial wellbeing manifests itself as a constant struggle to make ends meet, being
overwhelmed by debt, having circumstances dictate one's approach to
money, the need to cut back on essentials for one's children, and a higher likelihood of experiencing material hardship -- running out or worrying about running out of food, not being able to afford medical treatment or a place to live, or having
utilities turned off \citep{cfpb2017financial}.\footnote{For more detailed descriptions about the consequences of
    financial hardship, see \citet{cfpb2017financial, mps2018building,
stepchange2017strengthening}.} All of these experiences can cause a decline in
physical and mental health, a loss of productivity, and an overall
deterioration in the quality of life.

In the long-term, the danger is that the situation is not only self-sustaining
but leads to a vicious circle that becomes increasingly hard to escape. This can happen purely because having to borrow for regular expenses like food and bills can lead to a situation where an increasing amount has to be spent on servicing debt, which increases monthly expenses and the borrowed amounts required to cover them. However, it could also happen because of impaired decision making. A
literature on scarcity documents that our minds tend to focus on what is
scarce and to neglect what is not, thus concentrating our mental resources towards where they are most needed but reducing cognitive bandwidth in other domains \citep{shah2012some, mullainathan2013scarcity,
haushofer2014psychology}. For instance, \citet{mani2013poverty} find that
low-income shoppers in New Jersey perform worse on cognitive tasks when first
promoted to think about their financial situation while the same prompts had no
effect for wealthier shoppers, and sugar cane farmers in India perform worse on
similar cognitive tasks shortly before the annual harvest (when money is
scarce) than shortly thereafter (when money is plentiful). There is also
evidence that scarcity might lower productivity in the present:
\citet{kaur2021financial} randomise the timing of wage payments to low-income
piece-rate manufacturing workers and find that workers that receive their wages
early and are thus no longer liquidity constrained make fewer mistakes and
increase their output by 7 percent. In a context where money is scarce, this mechanism can lead to a situation where individuals have less cognitive bandwidth to think about questions that could help them escape the situation -- how to reduce spending, build the skills required for a higher-paying job, or transition to a healthier lifestyle that would increase physical and mental wellbeing and performance. Low financial wellbeing in the present might thus beget even lower financial wellbeing in the future.

The determinants of financial wellbeing are a combination of circumstantial and
external factors as well as the capabilities, beliefs, and behaviours of the
individual. \citet{mps2018building} provides a useful categorisation that sees
financial wellbeing as a function of four broad factors: (i) external factors that
include economic conditions, demographics, and a person's social environment; (ii)
enablers that include financial confidence and numeracy, a sense of control,
one's spending and savings mindset, and one's engagement with money, advice, and
technology; (iii) day-to-day behaviours like managing the use of credit, avoiding to
borrow for everyday spends, active saving, keeping track of and making
adjustments to one's spending, and shopping around; and, (iv), planning ahead
behaviours like building financial resilience through saving and planning for
retirement.

While no one factor is deterministic for financial wellbeing, research from the
US \citep{cfpb2017financial} and the UK \citep{mps2018building} agrees that -- as
expected -- external economic factors such as access to education and higher
paying jobs are important. In the UK, for instance, individuals with an annual
income of \pounds20,000 or below account for 41 percent of the working-age
population but for 69 percent of those with less than \pounds100 of savings.
Similarly, 50 percent of people who borrow to cover everyday expenses earn less
than \pounds17,000. So, clearly, a search for and support of effective economic
and social policy measures should be an important part of any effort to improve
societal financial wellbeing.

But the same research also shows that higher
incomes are not sufficient. In the UK, 18 percent of individuals with less than
\pounds 100 in savings have a household income of \pounds30,000 or higher, and
20 percent of those who borrow to cover everyday expenses have an income of
\pounds50,000 or higher. In the US, too, there is large variation in the
characteristics of individuals at each level of financial wellbeing; the
financial wellbeing of the top quarter of people with a high-school degree, for
instance, is higher than that of the bottom half of those with graduate
degrees.

In both countries, research shows that the level of savings is a key
contributor to financial wellbeing. In the US, it is the one factor that
discriminates between different levels of financial wellbeing better than any
other factor examined in \citep{cfpb2017financial}. In the UK, it is -- together
with behaviour towards credit -- the strongest predictor of financial wellbeing
\citep{mps2018building}. In fact, the same research suggests that having
\pounds1,000 in liquid savings could reduce the probability of being in debt by
almost half. In particular, having a habit of saving regularly -- even more so
than the amounts saved -- has been found to be a key determinant. Other
factors that are positively associated with higher financial wellbeing are
confidence in one's ability to achieve one's financial goals, not using debt to
cover everyday expenses, paying one's bills on time, staying within one's budget
and spending plan, paying credit card balances in full, and checking bank
statements for errors.

But many people struggle with these behaviours. For instance, in the UK, 21
percent of the working-age population (10.7 million adults) report to rarely or
never save, and 22 percent of the population have less than \pounds100 in
savings, with the proportion of those holding that amount in a formal savings account being even
lower \citep{mps2018building}. Unsurprisingly, then, one in four UK adults could
not pay an unexpected bill of \pounds300 from their own money
\citep{phillips2021supporting}. In the US, the situation is similar: 30 percent
of adults report that they would be unable to cover a bill of \$400 in cash or
its equivalents \citep{fed2022economic}, and, according to a 2013 survey,
liquid net worth for the median household with a head aged 41-51 is only \$813,
whereas at the 25th percentile it is -\$1,885
\citep{beshears2018behavioral}.\footnote{Liquid net worth includes all assets
    (except pension wealth, retirement savings, homes, and durable assets) net
of all debt (except student loans and collateralised debt).}

This is problematic because many households do face unexpected financial shocks
over the course of a year. Also, many households experience high income
volatility: in the US between 2013 and 2018, the median month-to-month change
in household income was 36 percent, with low-income households experiencing
more frequent and larger income dips \citep{jpmorgan2019weathering}. The same
research finds that families need roughly six weeks of take-home income
in liquid assets to weather an income dip and a simultaneous expenditure shock, but that 65
percent of households lack such a buffer. In line with these findings,
\citet{roll2020income} find that during the coronavirus pandemic, households
with liquid asses of above \$2,000 had significantly lower risk to experience indicator of financial distress such as skipping essential bills, being behind on
credit card debt, and being in overdraft than households with lower savings.

Managing debt is similarly challenging for many. 9 million people in the UK
also borrow to cover expenses for food and bills \citep{mps2018building}, and
many struggle to stay on top of their credit card debt: 2 million cards were in
arrears or default, another 2 million carried persistent debt, and for another
1.6 million cards, owners were persistently making minimum payments only.
Altogether, \citet{fca2016credit} finds that 5 million accounts (9 percent of the total)
that were active in January 2015 would, under their current repayment pattern
and without further borrowing, take 10 years or more to repay their balance. In addition to holding high and persistent balances, a
large body of research also indicates that individuals make other mistakes in
debt management, especially in dealing with credit cards: they choose
suboptimal credit card contracts \citep{agarwal2015consumers}, are overly susceptible to teaser rates \citep{shui2004time, ausubel1991failure}, rely on very-high-interest payday loans before exhausing lower-interest credit card limits \citep{agarwal2009payday} pay down debt across different cards proportionally to outstanding balances instead of prioritising
high-interest cards \citep{gathergood2019individuals}, hold
credit card debt and liquid assets at the same time \citep{gross2002liquidity,
    gathergood2020co}, and are overly influenced by stated minimum payments \citep{sakaguchi2022default}.\footnote{See \citet{agarwal2017shapes} for a more
    complete review of a large body of research documenting consumer choice
inefficiencies and suboptimal financial behaviour.} 


\section{Behavioural science}

Behavioural science can help address these challenges by understanding the
factors that determine financial behaviour and by designing and testing
possible solutions. The main factors that have been found to influence
financial decision making are: time-preferences and present bias
\citep{laibson1997golden, frederick2002time, read2018intertemporal,
ericson2019intertemporal, cohen2020measuring}, inertia
\citep{madrian2001power}, over-extrapolation \citep{choi2009reinforcement},
limited self-control and willpower \citep{thaler1981economic,
benhabib2005modeling, fudenberg2006dual, loewenstein2004animal,
gul2001temptation}, cognitive limitations and financial literacy
\citep{agarwal2009age, agarwal2013cognitive, korniotis2011older,
agarwal2010learning, fernandes2014financial, jorring2020financial}, attitude
towards money and spending \citep{rick2008tightwads, rick2011fatal}, one's
perceived locus of control \citep{perry2005control}, degree of optimism
\citep{puri2007optimism}, the ability to frame decisions broadly rather than
narrowly \citep{kumar2008decision}, propensity to gamble
\citep{kumar2009gambles}, one's social network \citep{bailey2018economic,
kuchler2021social}, the degree of one's financial planning
\citep{ameriks2003wealth}, and habits \citep{blumenstock2018defaults,
schaner2018persistent, de2013deposit}.\footnote{For two thorough reviews of the
literature, see \citet{agarwal2017shapes} and \citet{greenberg2019financial}.}

Researchers have also developed and tested a large number of approaches
designed to help people make better decisions. I briefly discuss four such approaches
that have the potential to help people make the financial choices they
themselves would like to make. A complete review of that literature is provided
by \citet{duckworth2018beyond}.

The arguably most studied and most successful approach is the use of defaults. Defaults are consequential because people often stick
with the status quo \citep{samuelson1988status}, tend to
interpret defaults as a recommendation \citep{mckenzie2006recommendations}, and stick with them even after resolving to do otherwise because of procrastination \citep{carroll2009optimal, ericson2017interaction}. They can also act as a reference point from which they are unwilling to move away from \citep{johnson2003defaults, kahneman1979prospect}. Default options have been
applied across range of areas and have, for instance, been found to increase
retirement savings contributions \citep{thaler2004save, madrian2001power,
beshears2009importance} and organ donations \citep{johnson2003defaults,
gimbel2003presumed, abadie2006impact}.

Another extensively studied approach is the commitment device, whereby an
individual restricts their future choice set in order to avoid choosing a
self-defeating action. While not everybody makes use of such devices when
offered the opportunity \citep{bryan2010commitment}, and while they do not work
in all contexts \citep{laibson2015don,robinson2018some}, they have been found
to help individuals increase their savings rates \citep{ashraf2006tying}, quit
smoking \citep{gine2010put}, make healthier food choices
\citep{schwartz2014healthier}, and exercise more regularly
\citep{royer2015incentives}. A popular implementation of a commitment device is
the platform stickk.com, which helps users
define their goals and leverages the power of loss aversion and accountability
to help people follow through with their desired behaviour change; it allows
users, for instance, to commit to donating money to a cause they do not support
if they fail to exercise as often as they wanted to, or to nominate a friend or
family member who will verify whether they managed to stick to their spending
goal.

Another approach are implementation intentions, a particular type of planning
for the achievement of one's goals that involves ``if-then'' intentions, such as
``if I get paid, then I transfer 10 percent of it into my savings account''
\citep{gollwitzer2006implementation, rogers2015beyond}. Such intentions have
been found to support perseverance in pursuing one's goals
\citep{oettingen2010strategies} and to increase overall goal attainment across
different age groups, life domains, and types of obstacles
\citep{gollwitzer2006implementation}.

A final intervention that has the potential to alter financial behaviour is
social norms messaging, whereby people are informed about how their own
behaviour compares to that of a relevant peer group. Such information can be
especially useful in domains where such information is usually not available,
as is the case with spending and saving. Such messaging has been successful in inducing high-energy-use households to lower their energy use without inducing low-use
households to increase theirs \citep{schultz2007constructive,
allcott2011social, allcott2014short, brandon2017effects}. However, the
information can also backfire and has been found to lower participation in
pension savings plans \citep{beshears2015effect} and completion rates of an
online course \citep{rogers2016discouraged}, making it important to test
messages before deploying them on a large scale to avoid unintended
consequences.

These approaches, together with many others, have the potential to make it
easier for individuals to manage their money and to spend and save in line with
their own goals. To realise that potential, however, they have to be tested on
a large scale and at low cost to gather reliable results and to have the
ability to tweak and refine them. Furthermore, solutions that are found to work
have to be deployed on a large scale to have a meaningful impact. The advent of
online and mobile banking and the large amount of data they create enable
researchers and financial institutions to do just that.


\section{FinTech}%

The combination of ever larger amounts of granular data about human behaviour, the ongoing progress in the development of machine-learning
algorithms to elicit information from such data, and the increase in computing power
that allows for the storage and processing of the data and the deployment of
these algorithms offers enormous potential for a better understanding of human
behaviour \citep{jaffe2014big, buyalskaya2021golden} and for building
applications that help people live in line with their financial goals.

Platforms like Facebook provide the opportunity to recruit study participants
at a scale and with an ease that was not possible before
\citep{kosinski2015facebook}, and data from its platform has been used to study the effect of social interactions on
house purchasing decisions \citep{bailey2018economic, bailey2019house}, the
spatial structure of urban networks \citep{bailey2020socialconnectedness}, the role of peer
effects in product adoption \citep{bailey2019peer}, the role of social networks
in determining social distancing during the coronavirus pandemic
\citep{bailey2020socialnetworks}, the determinants of social mobility
\citep{chetty2022sociali, chetty2022socialii}, and to predict participant personalities and life outcomes \citep{youyou2015computer}.

Log and sensor data from mobile phones are another increasingly popular data source in the study of human behaviour and have been used to successfully predict personality traits
\citep{montjoye2013predicting, stachl2020predicting} and levels of wealth
\citep{blumenstock2015predicting}, to improve human aid targeting
\citep{aiken2022machine}, and to study the role of social networks for migration
patterns \citep{blumenstock2019migration}. Some studies also combine data from
a range of sources; in one ambitious project, \citet{chi2022microestimates} use
data from satellites, mobile phone networks, topographical maps, and aggregated
and de-identified connectivity data from Facebook to map levels of wealth and
poverty at a 2.4km resolution for all 135 low and middle-income countries.

In the area of household finance, the advent of online and mobile banking is
creating an ever-growing amount of transaction-level data and a range of
FinTech apps aimed to help people manage their money. There are three ways in
which such data and apps act as critical enablers for an improved understanding
of financial behaviour and for the development, testing, and deployment of
possible solutions that help increase financial wellbeing.

First, large-scale and granular transaction data are an invaluable resource to
help researchers refine their understanding of the behavioural determinants of
financial decision making because they provide a more complete view of an individual's financial life, span a longer time
horizon, cover a larger set of individuals, and can be collected automatically, more reliably, and at a lower cost. Accordingly, such data have already been used to study a range of issues. For instance, \citet{kuchler2020sticking}
show that a failure to stick to self-set debt paydown schedules is best
explained by individual's present biasedness, \citet{gelman2014harnessing,olafsson2018liquid} show that consumer spending
varies across the pay cycle, \citet{baker2018debt,baugh2014disentangling} study
consumer spending responses in response to exogenous shocks,
\citet{carlin2019generational} document generational differences in financial
platform use, \citet{ganong2019consumer} show that consumer spending drops
sharply after the predictable income drop from exhausting unemployment
insurance benefits, \citet{meyer2018fully} analyse how individuals reinvest
realised capital gains and losses, and \citet{muggleton2020evidence} show that
chaotic spending behaviour is a harbinger of financial distress.\footnote{For a
    comprehensive review of the literature using financial transaction data,
see \citet{baker2022household}.}

Second, the FinTech apps from which such data are often gathered allow for
testing and the gradual refinement of possible solutions at high speed and
relatively low cost. Furthermore, because these apps are often built to help
users manage their money, they have a clear incentive to do so, something that
is not true to the same extent for more traditional banks. Some work that tests
the effectiveness of FinTech apps aimed to facilitate money management has
already been conducted: \citet{gargano2021goal} find that an app that allows
users to set savings goals does indeed increase savings, and
\citet{levi2020mind, carlin2022mobile} find, respectively, that improved access
to financial information reduces discretionary spending and non-sufficient fund
fees.

But there is large potential for further work in this area. For instance, while
the work of \citet{levi2020mind} suggests that lowering the cost of access to
financial information can improve financial outcomes, little is known about
what design features of mobile apps are particularly helpful. There are also
features that are not currently in widespread use that could address known
biases. One example is our tendency to underestimate the power of exponential
growth. \citet{mckenzie2011misunderstanding} find that highlighting to
participants the effects of exponential growth motivates them to save more. It
would be easy for mobile apps to project accumulated future savings for
different monthly savings amounts, say, and thus make the future loss from
spending more money in the present more salient. Another example is mental
accounting. \citet{soman2011earmarking} find that earmarking part of peoples'
salaries as savings increases savings rates. FinTech apps could, for instance,
suggest to automatically transfer a fixed amount into a labelled savings pot, thus
making that money less likely to be spent later. Finally, increasingly
sophisticated ``robo-advisers'' -- financial advice provided by
machine-learning algorithms -- has the potential to democratise personalised
financial advice across an increasing number of domains, from portfolio
allocation to consumption decisions to debt management
\citep{philippon2019fintech, dacunto2021new}.

In testing and refining possible solutions, the large size of the data is
critical because it allows for the identification of heterogeneous effects and
allows for the appropriate customisation of solutions. Having a view of users'
entire financial lives -- as in the case of financial aggregator apps that
allow users to add accounts from all their different financial providers -- is
also critical, since it allows for the identification of unintended
consequences; for example, inducing credit card users to make higher automatic
card repayments may increase automatic repayments but leave the total repayment
amount unchanged \citep{guttman2021semblance}, and auto-enrolling employees
into pension schemes might increase their debt levels
\citep{beshears2022borrowing}.

Finally, FinTech apps will make it relatively simple and cheap to make
effective solutions and tools widely available to a large audience, which is
critical to the goal of increasing financial wellbeing on a large scale.

Like all innovations, ever-growing data and increasingly sophisticated algorithms to study them also bring a new set of challenges. They can, for instance, compromise individual privacy and propagate
social biases. Inevitably, technological progress has largely outpaced
regulatory responses, and it will take time to understand all the implications
of these new technologies and enshrine proper safeguards into law. In the
meantime, leading researchers in the field are charting the course in thinking
about how to work with these data and tools responsibly
\citep{demontjoye2015unique, kosinski2015facebook, blumenstock2018don}, and
private companies as well as research institutions have put in place
safeguarding procedures to ensure that such best practices and current
regulations are followed at the level of individual research projects.


\section{Thesis outline}%

The thesis consists of three stand-alone chapters, each of which addresses a
different aspect of the overall aim of the thesis -- to combine large-scale
transaction data, insights from behavioural science, and methods from
econometrics and machine learning to study how individuals spend and save. 

Chapter~\ref{cha:mlbt} introduces and tests a new method that allows
researchers to elicit for individuals in large datasets behavioural traits of the type used in lab and survey
studies. This work contributes to the overall
thesis purpose indirectly in that it aims to augment the information contained
in large datasets with additional information that opens up additional avenues
of research and helps researchers and FinTech apps to fully leverage the data in their study of
human behaviour and the development, testing, and deployment of tools to help people reach their financial goals.

Chapter~\ref{cha:entropy} tests whether the way individuals spend their money
is related to how frequently they make transfers into their emergency savings
funds. The overall aim of the chapter is to test whether simple summary
statistics based on spending profiles -- the way in which individuals
allocate their spending transactions across different product and merchant
categories -- capture information about individuals' life circumstances that
are predictive of their financial behaviour. If so, such statistics could be
used in financial management apps to, for instance, provide customised assistance in time of need.

Chapter~\ref{cha:eval} tests whether using Money Dashboard -- the financial
aggregator app that provides the data for all chapters of this thesis -- is
associated with a reduction in discretionary spend and an increase in emergency
savings. The aim is to contribute to a better understanding of the extent to
which the potentially large benefits of FinTech apps already help users achieve
their financial goals and thus raise financial wellbeing.
