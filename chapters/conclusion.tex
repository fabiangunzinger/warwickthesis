% !TEX root = ../thesis.tex

\chapter{Conclusion}%
\label{cha:conclusion}

This thesis consists of three stand-alone chapters, each of which addresses a
different aspect of my overall aim of using large-scale transaction level data,
insights from behavioural science, and methods from econometrics and machine
learning to study how individuals spend money and save for the short-term.

Chapter~\ref{cha:mlbt}, which is joint work with Neil Stewart, introduces a
novel approach to elicit time preferences from mass financial transaction data.
Lab and field studies typically elicit time preferences by making participants
choose between a smaller-sooner and a larger-later effort or payoff. The core
idea of our approach is to identify a series of consumption choices that mimic
this trade-off, and then exploit these choices to estimate time preferences not
only for the subset of users for whom we observe the choice in the data but for
all users in the data.

The three choice scenarios we identify are monthly or yearly payment for car
insurance, purchasing a new mobile phone outright or paying in monthly
instalments, and swapping car and home insurer frequently or infrequently.
Each of these choices implicitly trades off a higher cost or effort in the
present but reduced cost in the future with lower cost or effort in the present
but higher-cost in the future. We identify individuals in our dataset that
engage in each of these choice scenarios, and then predict their likelihood of
making the patient choice based on features we extract from the data. We call
that likelihood individual's propensity for patience. We then use this
predictive model to predict the propensity for patience also for those users in
our dataset who didn't engage in a particular choice scenario. Finally, we test
whether predicted propensities from difference choice scenarios correlate
positively, as we would expect if they all elicit the same underlying time
preferences.

The main appeal of the approach is that, if successful, it would allow
researchers to estimate time-preferences for a large set of individuals based
on the smaller-sooner and larger-later payoff approach that researchers
commonly use in lab experiments. More broadly, the general approach we
introduce -- identify scenarios that identify choices used to estimate
behaviour traits, train a prediction model on individuals in the data for whom
that scenario is observed, and use that model to predict the trait for all
individuals in the data -- could, in principle, be used for other behaviour
traits, too. Large-scale observational data of the type used in this study has
become available only in recent years and is increasingly used across the
social sciences. Once challenge with such data is that they do not usually
provide the same detailed information about behaviour traits that can be gained
from observing a small set of people in a lab. Our approach has the potential
to bridge that divide by allowing researchers to augment large-scale
observational data with behaviour traits elicited from the data itself.

We find, however, that propensities based on different choice scenarios are
uncorrelated, suggesting that in our particular application, our approach is
unsuccessful at eliciting underlying preferences. There are three ways in which
our work could be refined in future research. First, identifying choice
scenarios is made more challenging because of imperfect tagging of transactions
in the data, and because we do not have direct access to the algorithms used
for tagging. Using the approach on data with more complete and reliable tagging
or in collaboration with the data provider, so that anomalies in the data can
be identified and handled, could eliminate that source of risk. Second, the
prediction models we used could be enhanced both with additional features and
by the use of more sophisticated models and further tuning. In terms of
features, we have already discussed that, in our context, adding variables that
more fully capture users' financial constraints might have been useful.
Furthermore, we have relied on a relatively straightforward set of features,
that could probably be augmented. In terms of models, we have used a number of
relatively simple ``off-the-shelf'' models. Using more advanced models could
help provide more accurate predictions. Finally, it is possible that people the
way people actually make intertemporal choices is not well captured, even
approximately, by the discounting models on which our elicitation approach
builds. If this is true, then either eliciting time-preferences based on a
different framework, or focusing on eliciting different behaviour traits might
prove more successful.

Chapter~\ref{cha:entropy}, also join work with Neil Stewart, calculates simple
summary statistics of individuals' spending profiles and tests whether
differences in spending profiles are predictive of the frequency with which
individuals make transactions into their savings accounts. To summarise
spending profiles, we calculate the entropy of an individual's spending profile
in a given period, which captures the predictability of spending transactions
in said period.

We focus on spending profiles for three reasons: first, our understanding of
how individuals spend their money is largely based on survey data from a
relatively small number of individuals. Large-scale transaction data of the
type used in this paper has become available to researchers only in recent
years and has not, to the best of our knowledge, been used to explore patterns
in consumer spending behaviour. Second, research suggests that spending
profiles might reflect circumstances in an individual's life that are also
related to their saving behaviour. Spending entropy, defined similarly as in
this chapter, has been found to predict financial distress, calorie intake, and
the frequency of and amount spend during supermarket visits. Finally, we think
of our focus on spending profile as a proof of concept for the broader agenda
of testing whether we can extract simple summary statistics from large-scale
transaction data that capture relevant information about individuals' life
circumstances. If successful, such information could be used in the design of
financial management apps to provide timely and customised assistance to users
when they need it most.

Our results show that spending entropy is predictive of the frequency of
savings transactions, with an effect size similar to that of an increase in
monthly income income of between \pounds1,000 and \pounds2,000. This holds true
if we control for simple component parts of entropy, suggesting that the
non-linear way in which entropy combines these components captures something
about spending profiles that is of relevance. The results are also largely
consistent across entropy measures calculated based on different product and
merchant categories. However, they direction of the effect changes for
different types of entropy that differ in the way in which we treat product or
merchant categories in which an individual makes no transactions in a given
period. Exploratory research suggests that the number of such zero count
categories, together with the variation of counts for categories with a
positive number of transactions, goes some way in explaining the outcome. But
more work is necessary to more fully understand this outcome, which we leave
for future research.

Chapter~\ref{cha:eval}, tests whether using Money Dashboard is associated with
a reduction in discretionary spending and an increase in emergency savings. I
use a new estimator proposed by \citet{callaway2021difference} that corrects
for recently identified problems in two-way fixed effects estimates.

I find that users reduce their discretionary spend by between \pounds100 and
\pounds150 (11-17\% of average discretionary spend) once they start using the
app and sustain that reduction throughout the six-month post-signup period I
consider. Looking at disaggregated measures of discretionary spend further
shows that the reduction is the result of maintained month-to-month changes in
behaviour rather than one-off cancellations of direct-debit transactions, that
it results from reducing spending on a number of different categories of
purchases rather than a single one, and that it is a result of changes along
the extensive rather than the intensive margin -- users reduce the number of
transactions they make rather than the value of the average transaction.
Interestingly, users do not seem to use these additional funds to build up
emergency savings: net-inflows into savings accounts do not change after
signup. I can also neither find significant increases in flows into investment
and pension accounts or additional savings accounts that are not linked to the
app, not additional loan repayments.

The main limitation of my approach is that I cannot isolate the effect of MDB
use from possible confounding factors that let users to self-select into using
the app in the first place, and that I cannot isolate the effect of individual
components of the app such as information and goal-setting. However, I do find
that app use is associated by a statistically and economically significantly
reducuction of discretionary spend that is sustained for at least six months,
and that seems to be brought about by a reduction in the number of purchases
across a number of spending categories. This provides novel insights into how
people who probably do have some unobserved motivation to reduce their savings
actually achieve this. The additional finding that a reduction in spend is not
accompanied by a commensurate increase in either short-term savings,
investments, or debt-reduction suggests that people use the saved funds for a
variety of different purposes.

The findings also suggest promising directions for further research. First, the
substantial drop in discretionary spend associated with app use does suggest
that MDB and apps like it might have a positive causal impact on financial
outcomes, making it worthwhile to study their effect in more detail in ways
that account for the two limitations mentioned above. Second, the finding that
people may direct saved funds towards a number of different purposes can be the
result of differences in funds use across or within people or a combination of
the two. Within people variation in fund use could suggest the possibility that people direct additional cash-flows
towards a number of different uses instead of towards those with the highest
payoff, such as paying down credit-card debt or investing in
government-subsidised high-interest investments. Further research establishing
to what extent there is within-person variation of fund use, and whether such
funds are directed towards the most effective uses would provide further useful
insights into savings behaviour, and could serve as the basis for further
development of financial aggregator apps to help users direct saved funds
towards the most effective users.


% %% limitations
% %% high level - we need other levers, too
% - Financial product design agarwal2017shapes
% - Regulation agarwal2017shapes
% - chater and loewenstein






