% !TEX root = ../thesis.tex

\chapter*{Abstract}%
\label{cha:abstract}
\addcontentsline{toc}{chapter}{\nameref{cha:abstract}}

Insights from behavioural science, if combined with FinTech apps and the data
these apps generate, offers the promise to make it vastly simpler for
individuals to manage their finances and, through that, to improve their
financial wellbeing. This thesis consists of three stand-alone chapters, all of
which aim to contribute to fulfilling that promise. They all use the same
dataset of financial transaction data from a large number of users of a
financial aggregator app. The first chapter introduces and tests a new approach
that allows researchers to elicit behaviour traits of the kind used in lab and
survey studies for individuals in large datasets. The approach allows
researchers to augment the information contained in large datasets and thus to
fully leverage such datasets for the study of human behaviour. The second
chapter tests whether the way individuals spend their money is related to how
frequently they make transfers into their emergency savings funds. The overall
aim of the chapter is to test whether simple summary statistics based on
spending spending profiles -- the way in which individuals allocate their
spending transactions across different product and merchant categories --
capture information about individuals' life circumstances that are predictive
of their financial behaviour. If so, such statistics could be used in financial
management apps, to, for instance, provide extra and customised assistance in
time of need. The third chapter tests whether using Money Dashboard -- the
financial aggregator app that provides the data for all chapters of this thesis
-- is associated with a reduction in discretionary spend and an increase in
emergency savings. The aim is to contribute to a better understanding of the
extent to which the potentially large benefits of FinTech apps already help
users achieve their financial goals and thus raise financial wellbeing.


