% !TEX root = ../thesis.tex

\chapter*{Abstract}%
\label{cha:abstract}
\addcontentsline{toc}{chapter}{\nameref{cha:abstract}}

Insights from behavioural science, if combined with FinTech apps and the data
these apps generate, offers the promise to make it vastly simpler for
individuals to manage their finances and, through that, to improve their
financial wellbeing. 

{\color{red} This thesis consists of three stand-alone chapters, all of which aim to contribute to fulfilling that promise. They all use the same
dataset of financial transaction data from a large number of users of a
financial aggregator app.}

{\color{blue} This thesis consists of three stand-alone chapters, all of which use the same dataset of financial transaction data from Money Dashboard, a financial aggregator app, and all of which aim to show that such data can gainfully employed in the study of consumer financial behaviour.}

The first chapter introduces and tests a new approach that allows researchers to elicit behaviour traits of the kind used in lab and
survey studies for individuals in large datasets. The approach allows
researchers to augment the information contained in large datasets and thus to
fully leverage such datasets for the study of human behaviour. {\color{blue} Our application of the approach to eliciting time preference is not successful, however, and we discuss possible reasons and make recommendations for future research.}

The second
chapter tests whether the way individuals spend their money is related to how
frequently they make transfers into their emergency savings funds. The overall
aim of the chapter is to test whether simple summary statistics based on
spending profiles -- the way in which individuals allocate their spending
transactions across different product and merchant categories -- capture
information about individuals' life circumstances that are predictive of their
financial behaviour. If so, financial management apps could use such statistics
to provide users with customised assistance in time of need, for instance. {\color{blue} The chapter specifically focuses on spending entropy, which captures the predictability of an individual's purchases. We show that spending entropy is indeed correlated of spending behaviour, and that the relationship is robust to different specifications and statistically and economically significant. We also show, however, that the direction of the relationship depends on the precise definition of entropy, and make recommendations for future research for how to investigate that fact further.}

The
third chapter tests whether using Money Dashboard is associated
with a reduction in discretionary spend and an increase in emergency savings.
The aim is to contribute to a better understanding of the extent to which the
potentially large benefits of FinTech apps already help users achieve their
financial goals and thus raise financial wellbeing. {\color{blue} We find that app usage is associated with a drop in discretionary spending of about \pounds150 during the first six months of app use, but that this is not mirrored by an increase in savings. We also show how the reduction in spending comes about, and discuss why we might not see a commensurate increase in savings.}
