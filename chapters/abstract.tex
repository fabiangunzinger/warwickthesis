% !TEX root = ../thesis.tex

\chapter*{Abstract}%
\label{cha:abstract}
\addcontentsline{toc}{chapter}{\nameref{cha:abstract}}
%There shall be a summary or abstract of the thesis, which should not exceed 300 words, bound in after the acknowledgements and declaration. The summary should not extend beyond a single A4 side, and to facilitate this, single spaced typing is permitted for the summary only. The summary shall provide a synopsis of the thesis and shall state clearly the nature and scope of the research undertaken and of the contribution made to the knowledge of the subject treated. There should be a brief statement of the method of investigation where appropriate, an outline of the major divisions or principal arguments of the work and a summary of any conclusions reached.


% Lara abstract
This thesis consists of three independent research studies in the fields of statis- tical and behavioural science. Each study is concerned with modelling complex spatio-temporal decisions recorded in police data. Analysing decisions at a high resolution requires a comprehensive understanding of the social phenomenon and data-generating mechanism, combined with careful modelling choices.
Chapter 1 is a novel model of ethnic bias at the officer-level in stop and search. Using a Bayesian hierarchical model, we model officer over-searching against two officer-specific baselines: the crime suspects that the officer encounters and the local patrolling area of the officer. We find that most police officers are biased against Black and Asian people in their search decisions, independently of which baseline we use. Furthermore, we decompose bias against ethnic minority groups into bias due to officer over-searching and over-patrolling.
Chapter 2 showcases the use of a spatio-temporal Hawkes-type point process to model the reporting of domestic abuse. Extending existing Hawkes models, we test for the existence of two spillover channels in crime victim reporting. Despite well-documented spillover effects in other human behaviour, we find no evidence to support such effects in the reporting of domestic abuse.
Chapter 3 introduces a new, robust statistical inference procedure for discrete outcomes. We propose using the Total Variation Distance together with Bayesian Nonparametric Learning to robustify inference. We show that this procedure pos- sesses a range of desirable theoretical properties. Furthermore, we demonstrate that our method outperforms standard inference both in terms of inference and out-of-sample performance on simulated data. Lastly, we show that robust infer- ence is important for modelling police-recorded incidence of sexual offences where fluctuations in reporting can drastically affect inference.
I conclude by discussing the importance of sophisticated statistical approaches to reflect often complicated underlying social phenomenon and the equally com- plex process by which it is recorded in data.
