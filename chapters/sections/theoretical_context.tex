% !TEX root = ../thesis.tex

\section{Theoretical context}%
\label{sec:theoretical_context}
% Explains the theoretical context and rationale for the inquiry underpinning the Thesis.


\subsection{Financial wellbeing and why it matters}%
\label{sub:financial_wellbeing_and_why_it_matters}

Financial wellbeing is a state in which a person can make ends meet in the
present, can feel comfortable about their financial future, can feel
comfortable about money, and has the financial freedom to make choices that
allow for the enjoyment of life. This includes having control over ones
day-to-day and month-to-month finances, not having to borrow to meet
ongoing obligations, being free of debt or being in control of it, having the
capacity to meet unexpected expenses, and -- for a working-age individual --
being on track to build enough savings for retirement.\footnote{The US Consumer
    Financial Protection Bureau defines fiancial wellbeing as ``a state of
    being wherein a person can fully meet current and ongoing financial
    obligations, can feel secure in their financial future, and is able to make
    choices that allow enjoyment of life'' \citep{cfpb2015financial}. The UK's
    Money and Pension service defines current financial wellbeing as ``being
    able to pay the bills and feel comfortable about money'' and longer-term
    financial security as ``having the savings to deal with the expected, the
    unexpected and the longer-term; or having some form of loss protection such
    as home contents or life insurance'' \citep{mps2018building}. Both
    organisations also differentiate some aspects of financial wellbeing
    depending on whether an individual is in their working or retirement age.
Throughout, this discussion, as well as in the analysis in the following
chapters, I focus on working age individuals.}

Financial wellbeing is important because a lack of it can have severe
consequences both in the short-term and in the long-term. In the short-term,
low financial wellbeing means a constant struggle to make ends meet, being
overwhelmed by debt, having circumstances dictate ones of and approach to
money, a higher likelihood of experiencing material hardship -- which
\citep{cfpb2017financial} defines as running out or worrying about running out
of food, not being able to afford medical treatment or a place to live, or have
utilities turned off -- and the need to cut back on essentials for ones
children.\footnote{For more detailed descriptions about the consequences of
    financial hardship, see \citet{cfpb2017financial, mps2018building,
stepchange2017strengthening}.} All of these experiences can cause a decline in
physical and mental health, a loss of productivity, and an overall deteriation
in the quality of life.

In the long-term, the danger is that the situation is not only self-sustaining
but leads to a viscious circle that becomes increasingly hard to escape. This
can happen purely because having to borrow for regular expenses like food and
bills -- as, the Money and Pension service estimates, about 9 million adults in
the UK do \citep{mps2018building} -- can lead to a situation where an
increasing amount has to be spent on servicing dept and thus increses monthly
expenses and the borrowed amounts required to cover them. In addition, however,
this could happen because of increasingly impaired decision making. A
literature on mental scarcity documents that our minds tend to focus on what is
scarce and neglect what is not, concentrating our mental resources where they
are most needed but reducing cognitive bandwith in other domains, which can
lead to poorer decision making.\footnote{See \citet{shah2012some,
    mullainathan2013scarcity, haushofer2014psychology} for excellent summaries
of the litearature.} For instance, \citet{mani2013poverty} find that low-income
shoppers in New Jersey perform worse on cognitive tasks when first promoted to
think about their financial situation while the same prompts had no effect for
wealthier shoppers, and sugarcane farmers in India perform worse on similar
cognitive tasks shortly before the annual harvest (when money is scarce) than
shortly thereafter (when money is plentiful). There is also evidence that
scarcity might lower preductivity in the present: \citet{kaur2021financial}
randomise the timing of wage payments to low-income piece-rate manufacturing
workers and find that workers that receive their wages early and are thus no
longer liquidity constrained make fewer mistakes and increase their output by 7
percent. Hence, if the anxiety caused by having little money to cover current
expenses lowers individuals' productivity in the present and makes it harder to
engage in rational planning for the future -- how to reduce spending, how to
build skills that would help find a higher-paying job, how to transition to a
healthier lifestyle that would help them feel and perform better physically and
mentally -- then low financial wellbeing in the present might beget low
financial wellbeing in the future.

The determinants of financial wellbeing are a combination of circumstantial and
external factors as well as the capabilities, believes, and behaviours of the
individual. \citet{mps2018building} provides a useful categorisation that sees
financial wellbeing as a function of four broad factors: external factors that
include economic conditions, demographics, and other external factors such as a
person's social environment; enablers that include financial confidence and
numeracy, a sense of control, ones spending and savings mindset, and ones
engagement with money, advice, and technology; day-to-day behaviours like
managing the use of credit, avoiding to borrow for everyday spends, active
saving, keeping track of and making adjustments to ones spending, and shopping
around; and, finally, planning ahead behaviours like building financial
resilience through saving and planning for retirement.\footnote{The US Consumer
Financial Protection Bureau uses a similar classification in its definition of
financial wellbeing \citep{cfpb2015financial}.}

While no one factor is deterministic for fiancial wellbeing, research from the
US \citep{cfpb2017financial} and the UK \citet{mps2018building} agrees that, as
expected, external economic factors such as access to education and higher
paying jobs are important. In the UK, for insteance, individuals with an annual
income of \pounds20,000 or below account for 41 percent of the working age
population but for 69 percent of those with less than \pounds100 of savings.
Similarly, 50 percent of people who borrow to cover everyday expenses earn less
than \pounds17,000. So, clearly, a search for and support of effective economic
and social policy measures should be an important part of any effort to improve
societal financial wellbeing. But the same research also shows that higher
incomes are not sufficient. In the UK, 18 percent of individuals with less than
\pounds 100 in savings have a household income of \pounds30,000 or higher, and
20 percent of those who borrow to cover everyday expenses have an income of
\pounds50,000 or higher. In the US, too, there ia large variation in the
characteristics of individuals at each level of financial wellbeing; the
financial wellbeing of the top quarter of people with a high-school degree, for
instance, is higher than that of the bottom half of those with graduate
degrees.

In both countries, research shows that the level of savings is a key
contributor to financial wellbeing. In the US, it is the one factor that
discriminates between different levels of financial wellbeing better than any
other examined factor \citep{cfpb2017financial}. In the UK, it is, together
with behaviour towards credit, the strongest predictor of financial wellbeing
\citep{mps2018building}. In fact, research by a UK charity suggests that having
\pounds1,000 in liquid savings could reduce the probability of being in dept by
almost half. In particular, having a habit of saving regulary -- even more so
than the amounts saved -- has been found to be a key determinant. Other
factors that are posivitely associated with higher financial wellbeing are
confidence in ones ability to achieve ones financial goals, not using debt to
cover everyday expenses, paying ones bills on time, staying withing ones budget
and spending plan, paying credit card balances in full, and checking bank
statements for errors.\footnote{For details on contributors to financial
wellbeing see \citep{cfpb2017financial} for the US and \citet{mps2018building}
for the UK.}

But many people struggle with these behaviours. For instance, in the UK, 21
percent of the working-age population (10.7 million adults) report to raraly or
never save, and 22 percent of the population have less than \pounds100 in
savings, with those holding that amount in a formal savings account being even
lower. Unsurprisingly, then, one in four adults could not pay an unexpected
bill of \pounds300 from their own money. In the US, the situation is similar,
with 30 percent of adults saying they would be unable to cover a bill of \$400.
This is problematic because many households do face unexpected financial shocks
over the course of a year. Also, many households experience high income
volatility: in the US between 2013 and 2018, the median month-to-month change
in household income was 36 percent, with low-income households experiencing
more frequent and larger income dips \citep{jpmorgan2019weathering}. The same
research finds that while families need roughly six weeks of take-home income
in liquid assets to weather a simultaneous income dip and expenditure shock, 65
percent of households lack such a buffer. In line with these findings,
\citet{roll2020income} find that during the coronavirus pandemic, households
with liquid asses of above \$2,000 had significantly lower risk to experience
financial distress (indicators like skipping essential bills, being behind on
credit card debt, being in overdraft) than households with lower savings.

Managing debt is similarly challenging for many. 9 million people in the UK
also borrow to cover expenses for food and bills \citep{mps2018building}, and
many struggle to stay on top of their credit card debt: 2 million cards were in
arrears or default, another 2 million carried persistent debt, and for another
1.6 million cards, owners were persistently making minimum payments only.
Altogether, the study found that 5 million accounts (9 percent of the total)
that were active in January 2015 would, under their current repayment pattern
and without further borrowing, take 10 years to repay their balance
\citep{fca2016credit}. In addition to holding high and persistent balances, a
large body of research also indicates that individuals make other mistakes in
debt management, especially in dealing with credit cards: they choose
suboptimal credit card contracts \citep{agarwal2015consumers}, sometimes
because they are overly susceptible to temporarily low teaser rates
\citep{shui2004time, ausubel1991failure}; they borrow on payday loans before
while being far from having exhausted their credit card limits
\citep{agarwal2009payday}, their repayment amount is overly influenced by
stated minimum payments \citep{sakaguchi2022default}, they pay down debt across
different cards proportionally to outstanding balances instead of prioritising
high-interest cards \citep{gathergood2019individuals}, and they sometimes hold
credit card debt and liquid assets at the same time \citep{gross2002liquidity,
    gathergood2020co}.\footnote{See \citet{agarwal2017shapes} for a more
    complete review of a large body of research documenting consumer choice
inefficiencies and suboptimal financial behaviour.} 

One promising way -- arguably the most promising way -- to address these
challenges and make managing money simpler is the use of insights from
behavioural science together with new technologies that collect data and can be
used to implement solutions.


\subsection{Role of behavioural science}%
\label{sub:role_of_behavioural_science}

Behavioural science can contribute to simpler money management and, as a
result, higher financial wellbeing because the discipline has accumulated a
substantial body of knowledge on the factors that determine financial behaviour
and solutions that could be used to address behaviours that lead to problematic
outcomes. The main factors that have been found to influence financial decision
making are: cognitive limitations and financial literacy \citep{agarwal2009age,
    agarwal2013cognitive, korniotis2011older, agarwal2010learning,
fernandes2014financial, jorring2020financial}; time-preferences and
self-control \citep{frederick2002time, read2018intertemporal,
ericson2019intertemporal, cohen2020measuring}; attitude towards money and
spending \citep{rick2008tightwads, rick2011fatal}; ones perceived locus of
control \citep{perry2005control}, degree of optimism \citep{puri2007optimism},
ability to frame decisions broadly rather than narrowly
\citep{kumar2008decision}, and propensity to gamble \citep{kumar2009gambles};
ones social network \citep{bailey2018economic, kuchler2021social}; the degree
of ones financial planning \citep{ameriks2003wealth}; and habits
\citep{blumenstock2018defaults, schaner2018persistent,
de2013deposit}.\footnote{For two thorough reviews, see
\citet{agarwal2017shapes} and \citet{greenberg2019financial}.}

Researchers have also developed and tested a large number of approaches aimed
to help people make better decisions. One main area of research here aims to
address limited self-control -- the difficulty most of us face at least
occasionally to act, moment-by-moment, in our own best interest and according
to our own goals. I briefly discuss four such approaches that have the
potential to help people make the fiancial choices they themselves would like
to make. A complete review of that literature is provided by
\citet{duckworth2018beyond}.

One extensively studied approach are commitment devices, whereby an individual
restricts their future choice set in order to avoid choosing a self-defeating
action. While not everybody makes use of such devices when offered the
opportunity \citep{bryan2010commitment}, and while they do not work in all
contexts \citep{laibson2015don,robinson2018some}, they have been found to help
individuals increase their savings rates \citep{ashraf2006tying}, quit smoking
\citep{gine2010put}, make healthier food choices \citep{schwartz2014healthier},
and exercise more regularly \citep{royer2015incentives}.

Another approach are implementation intentions, a particular type of planning
for the achievement of ones goals that involves ``if-then'' intentions, such as
``if I get paid, then I transfer 10 percent of it into my savings account''
\citep{gollwitzer2006implementation, rogers2015beyond}. Such intentions have
been found to support perseverance in pursueing ones goals
\citep{oettingen2010strategies} and to increase overall goal attainment across
different age groups, life domains, and types of obstacles
\citep{gollwitzer2006implementation}.

A third intervention that has the potential to alter financial behaviour is
social norms messaging, whereby people are informed about how their own
behaviour compares with that of a relevant peer group. Such information can be
especially useful for domains where such information is usually not available,
as is the case with spending and saving, and has been successful inducing high
energy use households to lower their energy use without inducing low-use
households to increase theirs \citep{schultz2007constructive,
allcott2011social, allcott2014short, brandon2017effects}. However, the
information can also backfire and has been found to lower participation in
pension savings plans \citep{beshears2015effect} and completion rates of an
online course \citep{rogers2016discouraged}, making it important to test
messages before deploying them on a large scale and to avoid unintended
consequences.

Finally, changing default options has been found to be a powerful tool for
behaviour change. Defaults are consequential becauase pepole often stick
with the status quo \citet{samuelson1988status}, and because they tend to
interpret defaults as a recommendation \citep{mckenzie2006recommendations},
implicitly view it as a reference point moving away from which would feel
costly \citet{johnson2003defaults, kahneman1979prospect}, and because even if
they resolve make a different choice, they often procrastiante
\citet{carroll2009optimal, ericson2017interaction}. Default options have been
applied across range of areas and have, for instance, been found to increase
retirement savings contributions \citep{madrian2001power,
beshears2009importance} and organ donations \citep{johnson2003defaults,
gimbel2003presumed, abadie2006impact}.

These approaches, together with many others, have the potential to make it
easier for individuals to manage their money and to spend and save in line with
their own goals. To realise that potential, however, we need to be able to test
these approaches on a large scale and at low cost to gather reliable results
and to have the ability to tweak and refine them, and we need to be able to
deply approaches that work on a large scale. The advent of online and mobile
banking and the large amount of data they create enable researchers and
financial institutions to do just that.


\subsection{Large scale data and fintech}%
\label{sub:large_scale_data_and_fintech}

The availability of very large amounts of very granular data about human
behaviour, together with ongoing progress in the development of
machine-learning algorithms elicit information from them, and the increase in
computer power that allows for the storage and processing of the data and the
deployment of these algorithms, offers enormous potential for better
understanding human behaviour \citep{jaffe2014big, buyalskaya2021golden} and
for building applications that help people live in line with their goals.

Platforms like Facebook provide the opportunity to recruit study participants
at a scale and with an ease that was not possible before
\citep{kosinski2015facebook}, and data from its platform has been used to
predict participant personalities and life outcomes \citep{youyou2015computer},
and to study the effect of social interactions on house purchasing decisions
\citep{bailey2018economic, bailey2019house}, the spatial structure of urban
networks \citep{bailey2020social}, the role of peer effects in product adaption
\citep{bailey2019peer}, the role of social networks in determining social
distancing during the coronavirus pandemic \citep{bailey2020social}, and the
determinants of social mobility \citep{chetty2022sociali, chetty2022socialii}.
Similarly, log and sensor data from mobile phones has been used to successfully
predict personality traits \citep{montjoye2013predicting,
stachl2020predicting}. 

In the area of household finance, the advent of online and mobile banking is
creating an every growing amount of transaction-level data and a range of
FinTech apps aimed to help people manage their money. Large-scale granular data
and widely-used mobile apps act as critical enablers for an improved
understanding of financial behaviour and the ability to test and deploy
solutions on a wide scale.

Transaction-level data from traditional financial institutions and FinTech apps
has already been used to study a range of issues. For instance,
\citet{kuchler2020sticking} show that a failure to stick to self-set debt
paydown schedules is best explained by individual's present biasedness,
\citet{gelman2014harnessing,olafsson2018liquid} show that consumer spending
varies across the pay cycle, \citet{baker2018debt,baugh2014disentangling} study
consumer spending responses in response to exogenous shocks,
\citet{carlin2019generational} document generational differences in financial
platform use, \citet{ganong2019consumer} show that consumer spending drops
sharply after the predictable income drop from exhausting unemployment
insurance benefits, \citet{meyer2018fully} analyse how individuals reinvest
realised capital gains and losses, and \citet{muggleton2020evidence} show that
chaotic spending behaviour is a harbinger of financial distress.\footnote{For a
comprehensive review of the literature using financial transaction data, see \citet{baker2022household}.}

In addition, FinTech apps have been used to learn what helps people manage
their finances. For instance, \citet{gargano2021goal} show that setting savings
goals in a fintech app increases savings, and \citet{levi2020mind} find that
individuals check their bank statement more frequently and reduce their
discretionary spend once they opt to use a mobile version of a financial
management app. There is large potential for futher work in this area. For
instance, while the work of \citet{levi2020mind} suggestst that lowering the
cost of access to financial information can improve financial outcomes, little
is known about what design features of mobile apps are particularly helpful.
There are also features that are not currently in widespread use that could
address known biases. One example is our tendency to underestimate the power of
exponential growth. \citet{mckenzie2011misunderstanding} find that highlighting
participants in a series of experiments the effects of exponential growth
motivates them to save more. It would be easy for mobile apps to project
accumulated future savings for different monthly savings amounts, say, and thus
make the future loss from spending more money in the present more salient.
Another example is mental accounting. \citet{soman2011earmarking} find that
earmarking part of peoples' salaries as savings increases savings rates.
Fintech apps could, for insteance, suggest to automatically transfer fixed
amount into labelled savings pot, thus making that money less likely to be
spent later.


- could even be default option - duckworth2018beyond for default lit


- robo advising (philippon2019fintech, dacunto2021frontiers)

- Challenges: what is good advise (sabat2019rules)

\citet{guttman2021semblance}
- Aggregator apps an prove particularly useful to study effect of intervention
because we can see complete financial picture, avoiding semblance of success
results. 

- Cheap and rapid iteration, and seeing full picture is necessary because (i)
interventions can have unintended consequences (semblance) and there is large
degree of heterogeneity both in financial wellbeing (lusardi?) and in the
response to interventions.




%%% privacy concerns
- \citet{demontjoye2015unique} find that it's easy to reidentify users based on
credit card metadata.

\citet{kosinski2015facebook} for discussion

