% !TEX root = ../thesis.tex

\section{Theoretical context}%
\label{sec:theoretical_context}
% Explains the theoretical context and rationale for the inquiry underpinning the Thesis.


%% Financial well-being matters

%%% Definitions:

\citet{mps2018building}
- Define current financial wellbeing as ability to pay bills and feeling
comfortable about money

- Longer-term financial security as having savings to deal with expected,
unexpected, and the long-term; or having some kind of loss protection such as
home contents or life insurance.


\citet{cfpb2015financial}
financial well-being can be defined as a state of being wherein a person can fully meet current and ongoing financial obligations, can feel secure in their financial future, and is able to make choices that allow enjoyment of life



%%% Determinants of financial well-being

\citet{cfpb2015financial}
- financial behaviours, financial knowledge, personal traits, social ane economic
environment, life stages (see report for good references)

\citet{cfpb2017financial}
- Relative role of different determinants of financial wellbeing.

- About a third of adults in US have score below 50 (out of 100), which is
associated with high p (>50pct) of struggling to make ends meet and experience
material hardship (running out or worrying about running out of food, not being
able to afford medical treatment or a place to live, or have utilities turned
off). -> scarcity 

- Disparities in financial well-being are greatest between individuals with
different levels of savings (among all examined factors). Savings are a key
ingredient to promote financial wellbeing. Further evidenced by narrow spread
of well being score within each subgroup of savings levels. 

- Confidence in achieving financial goals, having regular savings habit,
engaging in effective day-to-day money managemnt (paying bills on time, staying
within budget or spending plan, paying cc balance in full, checking statements
for errors) are all associated with higher fin well-being. -> data and tech can
really help here!

- Large variations within each subgroup suggests that no one factor is
determining of its own (e.g. top quarter in fin wb with high school degree have
well-being score that is higher than taht of bottom half of whose with graduate
degree).


%%% Facts:

\citet{mps2018building}
- Survey of 6000 UK adults, focused on understanding financial wellbeing and
capability

- Strongest predictors of FW are savings and credit behaviours

- One key measure is active saving - saving regularly more than amount - to
build savings habit

- 21 pct of UK population (10.7 mn adults) rarely or never save.

- 22 pct (11.5mn) have less than gpb 100 in savings. Those who hold that amount
in formal savings account is even lower.

- Similarly: from philipps2021supporting: 1 in 4 could not pay unexpected bill
of gpb 300 from their own money.

- Highly problematic, given that many households face unexpected financial
shock during the year.

- For extremely low savings, <100 gpb, income is the main driver, but not the
only one: attitudes and believes matter. But income is main point, connect to
chater2022iframe

- 9mn / 17pct borrow to cover food or bills. Thereof, 50pct have income < 17k,
but 20pct with >50k. Again, not only about money.

- Key contributors to financial wellbeing:

    - Not debt-to-income ratio as much
    - Use of credit to cover day-to-day expenses (I don't look at this)

- Retirement planning: among working age, 14 pct report to have done
substantial amount of retirement planning, 44 pct at least a good amount. Among
those between 45-65, this is 47 percent. 20 pct of that latter group have no
plan at all. 

- Again, income is important, but so is attitudes: know enough, confident to
plan their future. MPS suggests need to build confidence. Also difficult
because retirement seems far in the future.  Simpler tools can
help with that! So can good visualisation to make future more salient. 




%%% For the present

- 9 in 10 parents in debt pay back on essentials for children to repay debt
stepchange2017strengthen

Experienced well-being

Buffering shocks
- roll2020income

\citet{jpmorgan2019weathering}
- Income volatility is high: 36 pct change in month-to-month income for median
level (based on admin data of over 6 mn customer checking accounts between 2013
and 2018)
- Low-income families experience more frequent and larger income dips
- Families need roughly six weeks of take-home income in liquid assets to
weather simultaneous income dip and expenditure spike.
- 65 pct of households lack sufficient cash buffer for this




%%% For the future - scarcity spirals


%%% Many hoseholds don't have it - it being financial well-being

Absorbing shocks:
- in the UK, 25 percent of adults would be unable to cover an unexpected
bill of \pounds300 \citep{philipps2021supporting}, while in the US, about 30
percent would be unable to cover a \$400 bill \citep{fed2022economic}

Debt management:
\citet{fca2016credit}
- 2mn in arrears or default, another 2mn carrying persistent debt, another
1.6mn making persistent minimum payments.
- 9 pct / 5mn cc accounts actice in Jan 2015 would on current repayment pattern
and without further borrowing take 10 years to repay balance.


%%% Moderate changes could make large difference?

\citet{stepchange2017strengthening}
- Having 1000 gbp in savings reduces p of being in dept by almost half. 


%%% The subject is understudied

Exception:
\citet{philipps2021supporting}
- Automatic payroll deduction seems to be valued by employees and has helped
people from the squeezed and struggling segments 




%% Behavioural science can help understand behaviour and suggest design changes

%%% First step: understand to what extent personal choices contribute to low
%%% problem

%%% Understand drivers and design solutions

%%% Learn from pension research


%%% Identify other levers and how they can be pushed (chater2022iframe). This
%%% is not what I do, but is crucial.

- mps2018building shows that while income is sole determinant of very complete
lack of saving (<100gbp), it's the main one (18 percent of those with <100 earn
more than 30k - entire figure shows that very low income are disproportionaly
represented among low income savers, suggesting it's a lot about income).
Supports chater2022iframe point: look for more powerful levers, too, in
addition to focusing on non-income determinants.

- Similar for borrowing for everyday expenses: 50pct < 17k, but 20pct > 50k:
it's not only about income, but income is a very large part of the picture.
(Suggests that it's definitely worth focusing on behaviour: even if low income
people did have higher incomes, if they were similar to present high income
people, many of them still would have low savings).

- As cfpb2015financial acknowledges, structural opportunities like economic
context, family wealth and connections, access to education, geographic
location all play a major role in fianncial well-being. But there is still a
role for helping people improve their lives as they are. The danger is that we
do nothing else, but that's a separate issue. 


%% Newly available data and fintech as critical enablers

%%% Data allows us to study behav at large scale (which is transforming
%%% soc science research more broadly)

- buyalskaya2021golden for opportunities of new data for social research

- stachl2020predicting show that using sensor and log data from smartphones to
predict peoples' Big Five personality traits is about as accurate as using
social media footprints.

- montjoye2013predicting also predict personality (big five) from mobile log
data and find predictions to be considerably better than random.



%%% Fintech allows for large-scale implementation of findings and for rapid and
%%% continual testing

\begin{itemize}
    \item Current applications: simpler dashboards (Monzo, Money Dashboard),
        robo advising (philippon2019fintech, dacunto2021frontiers)

    \item Challenges: what is good advise (sabat2019rules)

    \item Adaptation in financial sector (boe2019machine)

    \item \citet{mckenzie2011misunderstanding} find that people systematically
        underestimate the exponential growth, and that showing them difference
        in long term outcome from regular savings left to compound makes them
        more motivated to save more for retirement. This is something fintech
        apps can easily help with by showing projections - or offering to show
        them - what is my wealth if I save x, or y, or z.

    \item \citet{soman2011earmarking} find that earmarking part of salary as
        savings increases savings rates. Fintech apps could suggest
        automatically transferring fixed amount into labelled savings pot.
\end{itemize}



%%% privacy concerns

\begin{itemize}
    \item \citet{demontjoye2015unique} find that it's easy to reidentify users
        based on credit card metadata.
\end{itemize}

