% !TEX root = ../thesis.tex

\section{Theoretical context}%
\label{sec:theoretical_context}
% Explains the theoretical context and rationale for the inquiry underpinning the Thesis.

%% Financial well-being matters


The question is important because a large number of adults in the UK and the US
do not have enough savings to cover unexpected expenses like car or medical
bills: in the UK, 25 percent of adults would be unable to cover an unexpected
bill of \pounds300 \citep{philipps2021supporting}, while in the US, about 30
percent would be unable to cover a \$400 bill \citep{fed2022economic}. But
while there is a large body of research that studies reasons for why savings
are low, little is known about what could help people save
more.\footnote{Well-documented behavioural biases that help explain undersaving
    are, among others, present bias \citep{laibson1997golden,
    laibson2019intertemporal}, inertia \citep{madrian2001power},
    over-extrapolation \citep{choi2009reinforcement}, and limited self-control
    and willpower \citep{thaler1981economic, benhabib2005modeling,
    fudenberg2006dual, loewenstein2004animal, gul2001temptation}. One danger of
    viewing low savings mainly as a result of behavioural biases is that while
    these biases likely do play some role and designing environments and tools
    to help correct them are thus part of the solution, it is at least
    conceivable that this is an area where the focus on behaviour-level
    solutions distracts from an effort to find more effective society-level
    solutions, a danger inherent in behavioural science research convincingly
    highlighted in \citet{chater2022frame}: if the main problem is that many
people are unable to earn enough to save, then the effectiveness of helping
them manage their low incomes more effectively pales in comparison with efforts
to help them earn more.}

\begin{itemize}
    \item Financial wellbeing is important.

    \item People in UK and US also don't have enough to cover unexpected
        outlays. (See reports). Also, \citet{sabat2019rules}

    \item This has important consequences:

        \begin{itemize}

            \item Short-term: financial well-being (see reports)

            \item Long-term (viscious cycle): scarcity hypothesis - makes it
                harder to focus on important things (plan for retirement, focus
                on healthy lifestyle, support children, ...) and might lead to
                vicious cycle (less savings leading to increased risk of
                financial hardship leading to more stress leading to less
                savings...)

            \item Buffering agains financial hardship \citep{roll2020income}
        \end{itemize}

    \item Spending and savings behaviour is important component - it's not all
        about lack of income.

\end{itemize}


