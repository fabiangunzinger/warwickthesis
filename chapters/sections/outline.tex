% !TEX root = ../thesis.tex

\section{Thesis outline}%
\label{sec:outline}

The thesis consists of three stand-alone chapters, each of which addresses a
different aspect of the overall aim of the thesis -- to combine large-scale
transaction data, insights from from behavioural science, and methods from
econometrics and machine learning to study how individuals spend and save. 

Chapter~\ref{cha:mlbt} introduces and tests a new method that allows
researchers to elicit behaviour traits of the type used in lab and survey
studies for individuals in large datasets. This work contributes to the overall
thesis purpose indirectly in that it aims to augment the information contained
in large datasets with additional information that opens up additional avenues
of research and helps researchers fully leverage the data in their study of
human behaviour.

Chapter~\ref{cha:entropy} tests whether the way individuals spend their money
is related to how frequently they make transfers into their emergency savings
funds. The overall aim of the chapter is to test whether simple summary
statistics based on spending spending profiles -- the way in which individuals
allocate their spending transactions across different product and merchant
categories -- capture information about individuals' life circumstances that
are predictive of their financial behaviour. If so, such statistics could be
used in financial management apps, to, for instance, provide extra and
customised assistance in time of need.

Chapter~\ref{cha:eval} tests whether using Money Dashboard -- the financial
aggregator app that provides the data for all chapters of this thesis -- is
associated with a reduction in discretionary spend and an increase in emergency
savings. The aim is to contribute to a better understanding of the extent to
which the potentially large benefits of FinTech apps already help users achieve
their financial goals and thus raise financial wellbeing.
