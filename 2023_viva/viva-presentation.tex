\documentclass[xcolor=svgnames]{beamer}
\usepackage{/Users/fgu/.dotfiles/latex/presentation}
\bibliography{/Users/fgu/dev/projects/dotfiles/latex/fabib}
\newcommand{\figdir}{}

\title{Essays in Consumer Financial Behaviour}

\author{Fabian Gunzinger \inst{1}}

\institute
{
    \inst{1}%
    University of Warwick\\
    \url{fabian.gunzinger@warwick.ac.uk}
}

\date {PhD viva \\ \vspace{1mm} 6 December 2023}

\begin{document}

\begin{frame}
  \titlepage
\end{frame}

\begin{frame}{The next 10 minutes...}
    \begin{itemize}
        \item Overall theme
        \item Chapter summaries
        \item Main findings and conclusions
    \end{itemize}
\end{frame}

\begin{frame}{Theme of thesis and context}

    Behavioural Science + FinTech = Financial Wellbeing

    \vspace{1cm}

    \textbf{Financial wellbeing}
    \begin{itemize}
        \item Many people struggle with it
        \item It has important short and long-term consequences
        \item Behavioural factors play an imporatnt role
    \end{itemize}
    \textbf{Behavioural science}
    \begin{itemize}
        \item Offers intervention with the potential to help
    \end{itemize}    
    \textbf{FinTech}
    \begin{itemize}
        \item Helps develop, refine, and test interventions
        \item Can implement successful interventions at scale
    \end{itemize}
\end{frame}

\begin{frame}{Context}

\end{frame}

\begin{frame}{Paper 1: Machine-learned time-preferences}
    \textbf{Objective}
    \begin{itemize}
        \item Introduce a novel approach to elicit behaviour traits from large-scale datasets
    \end{itemize}
    \textbf{Contribution}
    \begin{itemize}
        \item Enrich large-scale datasets with behavioural traits
        \item Applicable to any trait
        \item Relevant data needed for subset of users only
    \end{itemize}
    \textbf{Discussion}
    \begin{itemize}
        \item Our application not fully successful
        \item Recommendations for future research
    \end{itemize}
\end{frame}

\begin{frame}{Paper 2: Spending entropy predicts savings behaviour}
    \textbf{Objective}
    \begin{itemize}
        \item Broad: summary statistics from data to study financial behaviour
        \item Narrow: test whether entropy is predictive of savings behaviour
    \end{itemize}
    \textbf{Contribution}
    \begin{itemize}
        \item To emerging literature using txn data to study financial bahaviour
        \item Building an understanding of determinants of short-term savings
        \item Demonstrating that summary statistics from data can help with studying financial behaviour
    \end{itemize}
    \textbf{Discussion}
    \begin{itemize}
        \item Entropy is predictive of savings behaviour
        \item Summary measures like it can be used to improve FinTech apps
    \end{itemize}
\end{frame}

\begin{frame}{Paper 3: Does Money Dashboard help users spend less and save more?}
    \textbf{Objective}
    \begin{itemize}
        \item Test whether MDB helps users spend less and save more
    \end{itemize}
    \textbf{Contribution}
    \begin{itemize}
        \item Understand potential of FinTech apps to increase financial wellbeing
        \item Contribute to an understanding of determinants of emergency savings
        \item Furthers understanding of how users reduce spending
    \end{itemize}
    \textbf{Discussion}
    \begin{itemize}
        \item MDB is associated with reduction in spend but, interestingly, not savings
    \end{itemize}
\end{frame}

\begin{frame}{Overall contribution}
    \textbf{Insights}
    \begin{itemize}
        \item Novel approach to enrich txn data for better research
        \item Shows that summary statistics from data can help predict user behaviour
        \item Richer understanding of how users reduce spending
        \item Demonstrates that reduction in spend $\neq$ increase in savings
        \item Findings together with promising behavioural interventions: untapped potential for apps
        \item Collaboration between FinTech and researchers win-win 
    \end{itemize}
    \textbf{Open and transparent science}
    \begin{itemize}        
        \item I truly aimed to make research fully reproducible and understandable
        \item I carefully thought about data preprocessing to create reliable data
        \item All data processing steps clearly described
        \item All code available on Github
    \end{itemize}
\end{frame}




\begin{frame}\frametitle{Questions}
    Thank you for your attention.
\end{frame}

\end{document}

